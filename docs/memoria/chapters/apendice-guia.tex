\chapter{Guía para el desarrollador: creación de juegos para BreakBrain}
\label{chap::guia}

\drop{E}{ste} apéndice pretende ser una guía completa para desarrolladores. A lo largo del mismo se detallará el funcionamiento interno del subsistema de juegos de BreakBrain y se explicará, de forma clara y utilizando ejemplos prácticos, el procedimiento de creación e integración de juegos de terceros en BreakBrain.

Tras la lectura de esta guía, el desarrollador estará en disposición de extender la funcionalidad de BreakBrain mediante sus propios juegos, ya sean monojugador o multijugador, posibilitando que cualquier usuario de la red social haga uso de ellos.

\section{Requisitos}

Antes de poder desarrollar un juego para BreakBrain existen algunos requisitos que el desarrollador debe satisfacer. Suponen la posesión de una serie de conocimientos, habilidades y herramientas necesarias para desarrollar juegos HTML5 2D compatibles con la plataforma.

\subsection{Conocimientos del desarrollador}

Para desarrollar juegos para BreakBrain es necesario:

\begin{itemize}
\item Conocimiento avanzado del lenguaje JavaScript.
\item Experiencia desarrollando con NodeJS.
\item Experiencia en desarrollo de gráficos 2D.
\item Conocimiento avanzado del canvas de \acs{HTML} y manejo fluido del \acs{DOM}.
\end{itemize}

Además, resulta recomendable contar con:

\begin{itemize}
\item Experiencia en el desarrollo de aplicaciones web de tiempo real.
\item Experiencia en el manejo de websockets.
\end{itemize}

\subsection{Herramientas de desarrollo}

En cuanto a las herramientas necesarias para desarrollar juegos para BreakBrain, encontramos las siguientes:

\begin{itemize}
\item Editor de texto plano: puede servir cualquier editor de texto, aunque es recomendable contar con algún editor avanzado que ofrezca resaltado de sintaxis y facilidades para la programación. Se recomienda GNU Emacs, pero puede utilizar otras soluciones, como VIM, Sublime Text, Gedit, etc.
\item Navegador web: necesario para depurar y probar los juegos en desarrollo. Resulta muy importante que el navegador que utilicemos disponga de buenas herramientas para facilitar la depuración, así como de un buen motor de JavaScript que permita sacarle todo el potencial gráfico a la máquina. Se recomienda utilizar Chromium/Chrome o Mozilla Firefox.
\item Terminal de línea de comandos para ejecutar NodeJS: este {\it runtime} es el utilizado para ejecutar la parte servidora de los juegos, por lo que resulta esencial disponer de él para asegurar el correcto funcionamiento del juego.
\end{itemize}

Aunque no es necesario, además se recomienda disponer de:

\begin{itemize}
\item Sistema Operativo de tipo UNIX: para el desarrollo y la ejecución de los juegos es altamente recomendable un entorno UNIX, como GNU/Linux, BSD o Mac OS X. 
\end{itemize}

\section{Tipos de juegos}


\subsection{Juegos monojugador}


\subsection{Juegos multijugador}


\section{El sistema de juegos de BreakBrain}

aqui se explica cómo va la cosa


\subsection{Componentes del sistema de juegos}


\subsection{Componentes de un juego}


\subsubsection{Parte servidora}


\subsubsection{Parte cliente}


\subsubsection{Recursos}


\section{Creación de un juego desde cero}

\subsection{Eligiendo una categoría}


\subsection{Creando la estructura básica del juego}


\subsection{Preparando los recursos}

\subsection{Desarrollando la parte servidora}

\subsection{Desarrollando la parte cliente}


\section{Publicación del juego en BreakBrain}









ljksdf
fs
df




aquí va un código

\begin{listing}[language=JavaScript, caption={Hola mundo en C}, label=code:hello]
var a = function(arg){
  console.log('hola' + arg); // Esto es un comentario
};
\end{listing}

