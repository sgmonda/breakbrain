% -*- coding: utf-8 -*-

\chapter{Agradecimientos}

Con estas humildes líneas quiero, en primer lugar, mostrarte mi gratitud por leer este documento. Llegar hasta aquí ha supuesto un gran esfuerzo, tanto en el estudio de la carrera de Ingeniería Informática como durante el tiempo invertido para desarrollar este proyecto. Sin embargo, nada habría sido posible sin la ayuda de todas esas personas que han estado ahí día tras día. Este documento es mi pequeño regalo para todos ellos.

Gracias a mi familia, porque me apoyaron desde el primer momento en el que decidí alejarme trescientos kilómetros de casa para estudiar en Ciudad Real. Han sido algo más de cinco años de universidad en los que han sucedido muchas cosas, buenas y malas. Siempre hemos estado unidos y eso es lo más importante. Su ayuda ha sido muy importante en la elaboración de este documento.

Gracias a María, el amor de mi vida, por hacer que todo tenga sentido. Haberla tenido a mi lado durante este tiempo ha sido fundamental.

Gracias a Jesús, profesor y director de este proyecto. Desde el primer curso de la carrera ha sido un ejemplo de calidad en la enseñanza, exigencia y motivación. Su confianza en mí ha sido uno de los mayores alicientes para llegar hasta aquí.

También quiero dar las gracias a todas los que me han ayudado de alguna forma y no he mencionado. Gracias a los que están a mi lado desde que empecé a estudiar Ingeniería Informática, mis inestimables compañeros y amigos de Ciudad Real. Por último, gracias de forma especial a los que, por desgracia, ya no están conmigo. José Luis, jamás olvidaré que fuiste tú quien me prestó el primer libro de medicina que utilicé para iniciar este proyecto. Ojalá pudiera devolvértelo.

Si el lector encuentra este texto de agrado y disfruta durante su lectura, por favor, tenga en cuenta que el mérito es de todos ellos.

Gracias.

%Dar gracias a :
% Jose luis algarra
% rebeca
% maria
% testers que menciono

\quoteauthor{Sergio} % sí, esto se firma


% Local Variables:
% coding: utf-8
% mode: latex
% TeX-master: "main"
% mode: flyspell
% ispell-local-dictionary: "castellano8"
% End:
